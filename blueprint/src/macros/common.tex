% In this file you should put all LaTeX macros and settings to be used both by
% the pdf version and the web version.
% This should be most of your macros.

% The theorem-like environments defined below are those that appear by default
% in the dependency graph. See the README of leanblueprint if you need help to
% customize this.
% The configuration below use the theorem counter for all those environments
% (this is what the [theorem] arguments mean) and never resets it.
% If you want for instance to number them within chapters then you can add
% [chapter] at the end of the next line.

%%%%%%% THEOREMS %%%%%%%%%
\newtheorem{theorem}{Theorem}[section]
\newtheorem*{theorem*}{Theorem}
\newtheorem{prop}[theorem]{Proposition}
\newtheorem{obs}[theorem]{Observation}
\newtheorem{lemma}[theorem]{Lemma}
\newtheorem{cor}[theorem]{Corollary}
\newtheorem{applemma}{Lemma}[section]

\theoremstyle{definition}
\newtheorem{definition}[theorem]{Definition}

\theoremstyle{remark}
\newtheorem{rmk}[theorem]{Remark}
\newtheorem*{eg}{Example}
\newtheorem{ex}{Exercise}
\newtheorem*{remark*}{Remark}
\newtheorem*{remarks*}{Remarks}
\newtheorem*{notation*}{Notation}
\newtheorem*{convention*}{Convention}


%%%%%%% Relations

\newcommand{\defeq}{=_{\mathrm{def}}}
\newcommand{\iso}{\cong}
\newcommand{\equi}{\simeq}
\newcommand{\retequi}{\unlhd}
\newcommand{\op}{\mathrm{op}}
\newcommand{\Nat}{\mathbf{N}}
% \newcommand{\co}{\colon}
\newcommand{\suchthat}{\,|\,}


%%% Morphisms

\newcommand{\id}{\mathsf{id}}
\newcommand{\yo}{\mathsf{y}}
\newcommand{\Arr}{\mathsf{Arr}}
\newcommand{\CartArr}{\mathsf{CartArr}}
\newcommand{\unit}{\, \mathsf{unit} \, }
\newcommand{\counit}{\, \mathsf{counit} \, }

%%%% Objects

\newcommand{\catC}{\mathbf{C}}
\newcommand{\catD}{\mathbf{D}}
\newcommand{\pshC}{\psh{\catC}}
\newcommand{\psh}[1]{\mathbf{Psh}(#1)} %% consider renaming to \Psh
\newcommand{\PSH}[1]{\mathbf{PSH}(#1)}
\newcommand{\set}{\tcat{set}}
\newcommand{\Set}{\tcat{Set}}
\newcommand{\FinSet}{\tcat{Set}_\mathrm{fin}}
\newcommand{\SET}{\tcat{SET}}
\newcommand{\cat}{\tcat{cat}}
\newcommand{\ptcat}{\tcat{cat}_\bullet}
\newcommand{\Cat}{\tcat{Cat}}
\newcommand{\ptCat}{\tcat{Cat}_\bullet}
\newcommand{\grpd}{\tcat{grpd}}
\newcommand{\Grpd}{\tcat{Grpd}}
\newcommand{\ptgrpd}{\tcat{grpd}_\bullet}
\newcommand{\ptGrpd}{\tcat{Grpd}_\bullet}
\newcommand{\Pshgrpd}{\mathbf{Psh}(\grpd)}
\newcommand{\PshCat}{\mathbf{Psh}(\Cat)}

\newcommand{\terminal}{\bullet}
\newcommand{\Two}{\bullet+\bullet}


%%%% Syntax

\newcommand{\Type}{\mathsf{Type}}
\newcommand{\Term}{\mathsf{Tm}}
\newcommand{\tp}{\mathsf{tp}}
\newcommand{\disp}[1]{\mathsf{disp}_{#1}}
\newcommand{\var}{\mathsf{var}}


\newcommand{\Fib}{\mathsf{Fib}}
\newcommand{\lift}[2]{\mathsf{lift} \, #1 \, #2}
\newcommand{\fiber}{\mathsf{fiber}}
\newcommand{\Interval}{\mathbb{I}}
\newcommand{\Lift}[2]{\mathsf{L}_{#1}^{#2}}

% % Greek
\newcommand{\al}{\alpha}
\newcommand{\be}{\beta}
\newcommand{\ga}{\gamma}
\newcommand{\de}{\delta}
\newcommand{\ep}{\varepsilon}
\newcommand{\io}{\iota}
\newcommand{\ka}{\kappa}
\newcommand{\la}{\lambda}
\newcommand{\om}{\omega}
\newcommand{\si}{\sigma}

\newcommand{\Ga}{\Gamma}
\newcommand{\De}{\Delta}
\newcommand{\Th}{\Theta}
\newcommand{\La}{\Lambda}
\newcommand{\Si}{\Sigma}
\newcommand{\Om}{\Omega}

% % Families
\newcommand{\setbn}[2]{\left\{\left. #1 \ \middle\rvert\ #2 \right.\right\}}
\newcommand{\seqbn}[2]{\left(\left. #1 \ \middle\rvert\ #2 \right.\right)}
\newcommand{\sqseqbn}[2]{\left[\left. #1 \ \middle\rvert\ #2 \right.\right]}
\newcommand{\append}{\mathbin{^\frown}}


% % Displayed categories

\newcommand{\dcatC}{\mathbb{C}}
\newcommand{\dcatD}{\mathbb{D}}
\newcommand{\dcatE}{\mathbb{E}}
\newcommand{\dcatF}{\mathbb{F}}

\newcommand{\dconst}[2][]{\mathop{\constfont{const}_{#1}}{#2}}

\renewcommand{\hom}{\mathsf{hom}}
\newcommand{\mor}[2]{\hom(#1,#2)}
\newcommand{\dmor}[3]{\hom_{#2}({#1},{#3})}
\newcommand{\dto}[3]{\ensuremath{{#1} \to_{#2} {#3}}}
\newcommand{\dob}[2]{\ensuremath{{#1}_{#2}}}
\newcommand{\compose}[2]{\ensuremath{#1}\circ {#2}}
\newcommand{\depeq}[1][*]{=_{#1}}

\newcommand{\f}{\bar{f}}
\newcommand{\g}{\bar{g}}
\newcommand{\h}{\bar{h}}
